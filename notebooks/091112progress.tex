\documentclass[]{article}
\usepackage{lmodern}
\usepackage{amssymb,amsmath}
\usepackage{ifxetex,ifluatex}
\usepackage{fixltx2e} % provides \textsubscript
\ifnum 0\ifxetex 1\fi\ifluatex 1\fi=0 % if pdftex
  \usepackage[T1]{fontenc}
  \usepackage[utf8]{inputenc}
\else % if luatex or xelatex
  \ifxetex
    \usepackage{mathspec}
  \else
    \usepackage{fontspec}
  \fi
  \defaultfontfeatures{Ligatures=TeX,Scale=MatchLowercase}
\fi
% use upquote if available, for straight quotes in verbatim environments
\IfFileExists{upquote.sty}{\usepackage{upquote}}{}
% use microtype if available
\IfFileExists{microtype.sty}{%
\usepackage{microtype}
\UseMicrotypeSet[protrusion]{basicmath} % disable protrusion for tt fonts
}{}
\usepackage[margin=1in]{geometry}
\usepackage{hyperref}
\hypersetup{unicode=true,
            pdftitle={R Notebook},
            pdfborder={0 0 0},
            breaklinks=true}
\urlstyle{same}  % don't use monospace font for urls
\usepackage{color}
\usepackage{fancyvrb}
\newcommand{\VerbBar}{|}
\newcommand{\VERB}{\Verb[commandchars=\\\{\}]}
\DefineVerbatimEnvironment{Highlighting}{Verbatim}{commandchars=\\\{\}}
% Add ',fontsize=\small' for more characters per line
\usepackage{framed}
\definecolor{shadecolor}{RGB}{248,248,248}
\newenvironment{Shaded}{\begin{snugshade}}{\end{snugshade}}
\newcommand{\AlertTok}[1]{\textcolor[rgb]{0.94,0.16,0.16}{#1}}
\newcommand{\AnnotationTok}[1]{\textcolor[rgb]{0.56,0.35,0.01}{\textbf{\textit{#1}}}}
\newcommand{\AttributeTok}[1]{\textcolor[rgb]{0.77,0.63,0.00}{#1}}
\newcommand{\BaseNTok}[1]{\textcolor[rgb]{0.00,0.00,0.81}{#1}}
\newcommand{\BuiltInTok}[1]{#1}
\newcommand{\CharTok}[1]{\textcolor[rgb]{0.31,0.60,0.02}{#1}}
\newcommand{\CommentTok}[1]{\textcolor[rgb]{0.56,0.35,0.01}{\textit{#1}}}
\newcommand{\CommentVarTok}[1]{\textcolor[rgb]{0.56,0.35,0.01}{\textbf{\textit{#1}}}}
\newcommand{\ConstantTok}[1]{\textcolor[rgb]{0.00,0.00,0.00}{#1}}
\newcommand{\ControlFlowTok}[1]{\textcolor[rgb]{0.13,0.29,0.53}{\textbf{#1}}}
\newcommand{\DataTypeTok}[1]{\textcolor[rgb]{0.13,0.29,0.53}{#1}}
\newcommand{\DecValTok}[1]{\textcolor[rgb]{0.00,0.00,0.81}{#1}}
\newcommand{\DocumentationTok}[1]{\textcolor[rgb]{0.56,0.35,0.01}{\textbf{\textit{#1}}}}
\newcommand{\ErrorTok}[1]{\textcolor[rgb]{0.64,0.00,0.00}{\textbf{#1}}}
\newcommand{\ExtensionTok}[1]{#1}
\newcommand{\FloatTok}[1]{\textcolor[rgb]{0.00,0.00,0.81}{#1}}
\newcommand{\FunctionTok}[1]{\textcolor[rgb]{0.00,0.00,0.00}{#1}}
\newcommand{\ImportTok}[1]{#1}
\newcommand{\InformationTok}[1]{\textcolor[rgb]{0.56,0.35,0.01}{\textbf{\textit{#1}}}}
\newcommand{\KeywordTok}[1]{\textcolor[rgb]{0.13,0.29,0.53}{\textbf{#1}}}
\newcommand{\NormalTok}[1]{#1}
\newcommand{\OperatorTok}[1]{\textcolor[rgb]{0.81,0.36,0.00}{\textbf{#1}}}
\newcommand{\OtherTok}[1]{\textcolor[rgb]{0.56,0.35,0.01}{#1}}
\newcommand{\PreprocessorTok}[1]{\textcolor[rgb]{0.56,0.35,0.01}{\textit{#1}}}
\newcommand{\RegionMarkerTok}[1]{#1}
\newcommand{\SpecialCharTok}[1]{\textcolor[rgb]{0.00,0.00,0.00}{#1}}
\newcommand{\SpecialStringTok}[1]{\textcolor[rgb]{0.31,0.60,0.02}{#1}}
\newcommand{\StringTok}[1]{\textcolor[rgb]{0.31,0.60,0.02}{#1}}
\newcommand{\VariableTok}[1]{\textcolor[rgb]{0.00,0.00,0.00}{#1}}
\newcommand{\VerbatimStringTok}[1]{\textcolor[rgb]{0.31,0.60,0.02}{#1}}
\newcommand{\WarningTok}[1]{\textcolor[rgb]{0.56,0.35,0.01}{\textbf{\textit{#1}}}}
\usepackage{graphicx,grffile}
\makeatletter
\def\maxwidth{\ifdim\Gin@nat@width>\linewidth\linewidth\else\Gin@nat@width\fi}
\def\maxheight{\ifdim\Gin@nat@height>\textheight\textheight\else\Gin@nat@height\fi}
\makeatother
% Scale images if necessary, so that they will not overflow the page
% margins by default, and it is still possible to overwrite the defaults
% using explicit options in \includegraphics[width, height, ...]{}
\setkeys{Gin}{width=\maxwidth,height=\maxheight,keepaspectratio}
\IfFileExists{parskip.sty}{%
\usepackage{parskip}
}{% else
\setlength{\parindent}{0pt}
\setlength{\parskip}{6pt plus 2pt minus 1pt}
}
\setlength{\emergencystretch}{3em}  % prevent overfull lines
\providecommand{\tightlist}{%
  \setlength{\itemsep}{0pt}\setlength{\parskip}{0pt}}
\setcounter{secnumdepth}{0}
% Redefines (sub)paragraphs to behave more like sections
\ifx\paragraph\undefined\else
\let\oldparagraph\paragraph
\renewcommand{\paragraph}[1]{\oldparagraph{#1}\mbox{}}
\fi
\ifx\subparagraph\undefined\else
\let\oldsubparagraph\subparagraph
\renewcommand{\subparagraph}[1]{\oldsubparagraph{#1}\mbox{}}
\fi

%%% Use protect on footnotes to avoid problems with footnotes in titles
\let\rmarkdownfootnote\footnote%
\def\footnote{\protect\rmarkdownfootnote}

%%% Change title format to be more compact
\usepackage{titling}

% Create subtitle command for use in maketitle
\providecommand{\subtitle}[1]{
  \posttitle{
    \begin{center}\large#1\end{center}
    }
}

\setlength{\droptitle}{-2em}

  \title{R Notebook}
    \pretitle{\vspace{\droptitle}\centering\huge}
  \posttitle{\par}
    \author{}
    \preauthor{}\postauthor{}
    \date{}
    \predate{}\postdate{}
  

\begin{document}
\maketitle

\hypertarget{code}{%
\section{Code}\label{code}}

\begin{Shaded}
\begin{Highlighting}[]
\KeywordTok{require}\NormalTok{(stringr)}
\KeywordTok{source}\NormalTok{(}\StringTok{"../shapevol1/R/sgene.R"}\NormalTok{)}
\KeywordTok{source}\NormalTok{(}\StringTok{"../shapevol1/R/genetostl.R"}\NormalTok{)}
\end{Highlighting}
\end{Shaded}

\hypertarget{genes-in-the-seed}{%
\section{Genes in the seed:}\label{genes-in-the-seed}}

The seed I'm working with is an adaptation of figure 4 in the paper, but
with a square section. Thus we can define \texttt{group1} in the seed
as:

\begin{Shaded}
\begin{Highlighting}[]
\NormalTok{gene01 <-}\StringTok{ }\KeywordTok{sgene}\NormalTok{(}\StringTok{"Cross section"}\NormalTok{,}\StringTok{"Square"}\NormalTok{,}\DataTypeTok{status =}\NormalTok{ T,}\DataTypeTok{start=}\OperatorTok{-}\DecValTok{40}\NormalTok{,}\DataTypeTok{stop=}\DecValTok{40}\NormalTok{,}\DataTypeTok{dom=}\DecValTok{1}\NormalTok{)}
\NormalTok{gene02 <-}\StringTok{ }\KeywordTok{sgene}\NormalTok{(}\StringTok{"Length"}\NormalTok{,       }\DecValTok{5}\NormalTok{       ,}\DataTypeTok{status =}\NormalTok{ T,}\DataTypeTok{start=}\OperatorTok{-}\DecValTok{40}\NormalTok{,}\DataTypeTok{stop=}\DecValTok{40}\NormalTok{,}\DataTypeTok{dom=}\DecValTok{1}\NormalTok{)}
\NormalTok{gene03 <-}\StringTok{ }\KeywordTok{sgene}\NormalTok{(}\StringTok{"Diameter"}\NormalTok{,     }\DecValTok{1}\NormalTok{       ,}\DataTypeTok{status =}\NormalTok{ T,}\DataTypeTok{start=}\OperatorTok{-}\DecValTok{40}\NormalTok{,}\DataTypeTok{stop=}\DecValTok{40}\NormalTok{,}\DataTypeTok{dom=}\DecValTok{0}\NormalTok{)}
\NormalTok{group1 <-}\StringTok{ }\KeywordTok{rbind}\NormalTok{(gene01,gene02,gene03)}
\end{Highlighting}
\end{Shaded}

First thing is to grow a leg. We've set the dominance :

\begin{Shaded}
\begin{Highlighting}[]
\NormalTok{gene04 <-}\StringTok{ }\KeywordTok{sgene}\NormalTok{(}\StringTok{"X_1Z"}\NormalTok{,          }\DecValTok{0}\NormalTok{,}\DataTypeTok{status =}\NormalTok{ T,}\DataTypeTok{start=}\DecValTok{0}\NormalTok{,}\DataTypeTok{stop=}\DecValTok{5}\NormalTok{,}\DataTypeTok{dom=}\DecValTok{50}\NormalTok{)}
\NormalTok{gene05 <-}\StringTok{ }\KeywordTok{sgene}\NormalTok{(}\StringTok{"Y_1Z"}\NormalTok{,          }\DecValTok{0}\NormalTok{,}\DataTypeTok{status =}\NormalTok{ T,}\DataTypeTok{start=}\DecValTok{0}\NormalTok{,}\DataTypeTok{stop=}\DecValTok{5}\NormalTok{,}\DataTypeTok{dom=}\DecValTok{50}\NormalTok{)}
\NormalTok{gene06 <-}\StringTok{ }\KeywordTok{sgene}\NormalTok{(}\StringTok{"Z_1Z"}\NormalTok{,          }\DecValTok{1}\NormalTok{,}\DataTypeTok{status =}\NormalTok{ T,}\DataTypeTok{start=}\DecValTok{0}\NormalTok{,}\DataTypeTok{stop=}\DecValTok{5}\NormalTok{,}\DataTypeTok{dom=}\DecValTok{50}\NormalTok{)}
\NormalTok{group2 <-}\StringTok{ }\KeywordTok{rbind}\NormalTok{(gene04,gene05,gene06)}
\end{Highlighting}
\end{Shaded}

\begin{Shaded}
\begin{Highlighting}[]
\KeywordTok{require}\NormalTok{(stringr)}
\KeywordTok{source}\NormalTok{(}\StringTok{"../shapevol1/R/sgene.R"}\NormalTok{)}
\KeywordTok{source}\NormalTok{(}\StringTok{"../shapevol1/R/genetostl.R"}\NormalTok{)}
\KeywordTok{genetostlfile3}\NormalTok{   (}\StringTok{"~/Desktop/stl/seedXg2.stl"}\NormalTok{,}\KeywordTok{rbind}\NormalTok{(group1,group2),}\DataTypeTok{runlim=}\DecValTok{10}\NormalTok{)}
\end{Highlighting}
\end{Shaded}

OK, That grows the leg - let's now add group 3 and see what happens.
This group of genes can make most of the top of the table:

\begin{Shaded}
\begin{Highlighting}[]
\NormalTok{gene13 <-}\StringTok{ }\KeywordTok{sgene}\NormalTok{(}\StringTok{"X_2Z"}\NormalTok{,         }\DecValTok{-1}\NormalTok{,}\DataTypeTok{status =}\NormalTok{ T,}\DataTypeTok{start=}\OperatorTok{-}\DecValTok{15}\NormalTok{,}\DataTypeTok{stop=}\DecValTok{10}\NormalTok{,}\DataTypeTok{dom=}\DecValTok{46}\NormalTok{)}
\NormalTok{gene14 <-}\StringTok{ }\KeywordTok{sgene}\NormalTok{(}\StringTok{"Y_2Z"}\NormalTok{,          }\DecValTok{0}\NormalTok{,}\DataTypeTok{status =}\NormalTok{ T,}\DataTypeTok{start=}\OperatorTok{-}\DecValTok{15}\NormalTok{,}\DataTypeTok{stop=}\DecValTok{10}\NormalTok{,}\DataTypeTok{dom=}\DecValTok{46}\NormalTok{)}
\NormalTok{gene15 <-}\StringTok{ }\KeywordTok{sgene}\NormalTok{(}\StringTok{"Z_2Z"}\NormalTok{,          }\DecValTok{0}\NormalTok{,}\DataTypeTok{status =}\NormalTok{ T,}\DataTypeTok{start=}\OperatorTok{-}\DecValTok{15}\NormalTok{,}\DataTypeTok{stop=}\DecValTok{10}\NormalTok{,}\DataTypeTok{dom=}\DecValTok{46}\NormalTok{)}

\NormalTok{gene16 <-}\StringTok{ }\KeywordTok{sgene}\NormalTok{(}\StringTok{"X_2X"}\NormalTok{,          }\DecValTok{1}\NormalTok{,}\DataTypeTok{status =}\NormalTok{ T,}\DataTypeTok{start=}\DecValTok{0}\NormalTok{,}\DataTypeTok{stop=}\DecValTok{8}\NormalTok{,}\DataTypeTok{dom=}\DecValTok{48}\NormalTok{)}
\NormalTok{gene17 <-}\StringTok{ }\KeywordTok{sgene}\NormalTok{(}\StringTok{"Y_2X"}\NormalTok{,          }\DecValTok{0}\NormalTok{,}\DataTypeTok{status =}\NormalTok{ T,}\DataTypeTok{start=}\DecValTok{0}\NormalTok{,}\DataTypeTok{stop=}\DecValTok{8}\NormalTok{,}\DataTypeTok{dom=}\DecValTok{48}\NormalTok{)}
\NormalTok{gene18 <-}\StringTok{ }\KeywordTok{sgene}\NormalTok{(}\StringTok{"Z_2X"}\NormalTok{,          }\DecValTok{0}\NormalTok{,}\DataTypeTok{status =}\NormalTok{ T,}\DataTypeTok{start=}\DecValTok{0}\NormalTok{,}\DataTypeTok{stop=}\DecValTok{8}\NormalTok{,}\DataTypeTok{dom=}\DecValTok{48}\NormalTok{)}

\NormalTok{gene19 <-}\StringTok{ }\KeywordTok{sgene}\NormalTok{(}\StringTok{"X_2Y"}\NormalTok{,          }\DecValTok{1}\NormalTok{,}\DataTypeTok{status =}\NormalTok{ T,}\DataTypeTok{start=}\DecValTok{0}\NormalTok{,}\DataTypeTok{stop=}\DecValTok{15}\NormalTok{,}\DataTypeTok{dom=}\DecValTok{48}\NormalTok{)}
\NormalTok{gene20 <-}\StringTok{ }\KeywordTok{sgene}\NormalTok{(}\StringTok{"Y_2Y"}\NormalTok{,          }\DecValTok{0}\NormalTok{,}\DataTypeTok{status =}\NormalTok{ T,}\DataTypeTok{start=}\DecValTok{0}\NormalTok{,}\DataTypeTok{stop=}\DecValTok{15}\NormalTok{,}\DataTypeTok{dom=}\DecValTok{48}\NormalTok{)}
\NormalTok{gene21 <-}\StringTok{ }\KeywordTok{sgene}\NormalTok{(}\StringTok{"Z_2Y"}\NormalTok{,          }\DecValTok{0}\NormalTok{,}\DataTypeTok{status =}\NormalTok{ T,}\DataTypeTok{start=}\DecValTok{0}\NormalTok{,}\DataTypeTok{stop=}\DecValTok{15}\NormalTok{,}\DataTypeTok{dom=}\DecValTok{48}\NormalTok{)}
\NormalTok{group3 <-}\StringTok{ }\KeywordTok{rbind}\NormalTok{(gene13,gene14,gene15,gene16,gene17,gene18,gene19,gene20,gene21)}
\end{Highlighting}
\end{Shaded}


\end{document}
